\documentclass[11pt]{article}
\usepackage{./reportNTU}
%\input{./macro_ihwang.tex}

\usepackage{tikz}
\usetikzlibrary{positioning, arrows}

\pagestyle{plain}

 
\rhead{}
\lhead{}
\chead{}
\lfoot{}
\cfoot{\thepage}
\rfoot{}

\newtheorem{theorem}{Theorem}
\newtheorem{corollary}[theorem]{Corollary}
\newtheorem{lemma}[theorem]{Lemma}

% macros
\newcommand{\argmin}{\mathop{\mathrm{argmin}}}
\newcommand{\argmax}{\mathop{\mathrm{argmax}}}
%\newcommand{\mathbbm}[1]{\text{\textbf{#1}}} % from mathbbm.sty
\newcommand{\norm}[2]{\left\lVert#2\right\rVert_{#1}}
\newcommand{\expect}[2]{\mathbb{E}_{#1}\left[#2\right]}
\newcommand{\derivative}[2]{\frac{\partial#1}{\partial#2}}
\newcommand{\inprod}[2]{\langle #1 , #2 \rangle}
\newcommand{\abs}[1]{\left \lvert #1 \right \rvert}
\newcommand{\Abs}[1]{\left \lVert {#1} \right \rVert}
\newcommand{\HDH}[0]{\mathcal{H} \Delta \mathcal{H}}
\newcommand{\DHDH}[0]{d_{\mathcal{H} \Delta \mathcal{H}}}
\newcommand{\EDHDH}[0]{\hat{d}_{\mathcal{H} \Delta \mathcal{H}}}
% \newcommand{\RiskT}[0]{R_{\mathcal{D}_T}}
% \newcommand{\RiskS}[0]{R_{\mathcal{D}_S}}
% \newcommand{\ERiskT}[0]{\hat{R}_{\mathcal{D}_T}}
% \newcommand{\ERiskS}[0]{\hat{R}_{\mathcal{D}_S}}
\newcommand{\RiskT}[0]{\epsilon_{T}}
\newcommand{\RiskS}[0]{\epsilon_{S}}
\newcommand{\ERiskT}[0]{\hat{\epsilon}_{T}}
\newcommand{\ERiskS}[0]{\hat{\epsilon}_{S}}
\newcommand{\DS}[0]{\mathcal{D}_S}
\newcommand{\DT}[0]{\mathcal{D}_T}
\newcommand{\tDS}[0]{\tilde{\mathcal{D}}_S}
\newcommand{\tDT}[0]{\tilde{\mathcal{D}}_T}
\newcommand{\US}[0]{\mathcal{U}_S}
\newcommand{\UT}[0]{\mathcal{U}_T}
\newcommand{\TUS}[0]{\tilde{\mathcal{U}}_S}
\newcommand{\TUT}[0]{\tilde{\mathcal{U}}_T}

\title{Title}
\author{Names}
\date{Student ID}
\begin{document}
\maketitle

\thispagestyle{plain}

\begin{abstract}
In this project, we focus on the problem of \textit{domain adaptation}, one of the transfer learning problems with the setting in which there are many labeled training data from a source domain, but no labeled testing data from a target domain. Specifically, we will formally state the domain adaptation problem and introduce the theoretical upper bound on the true target risk established in the literature. The upper bound is stated in terms of the source risk and a classifier-induced divergence measure between the training (source) and testing (target) domains, which formalizes the intuition that a good feature representation for cross-domain transfer must be discriminative for the main learning task on the source domain, while at the same time be indiscriminate with respect to the shift between the two domains. Based on the above theoretical principles, we then introduce the domain-adversarial neural network (DANN) that implements this idea in the context of neural network architectures and shows decent empirical results in many domain adaptation problems.
\end{abstract}

\section{Introduction}\label{sect:intro}
In the basic setting of PAC learning framework discussed in Lecture 1, 2, and 3, it is assumed that the training and testing data are drawn i.i.d. from the same distribution $P$. Under this basic setting, we can derive generalization performance guarantees of \textit{empirical risk minimization} (ERM) learning rule by developing several upper bounds on the estimation errors (or their tail probabilities), as well as the corresponding sample complexities.

In a more practical setting, however, training and testing data come from different \textit{domains}, namely, different data distributions. In the \textit{domain adaptation} problems, one faces machine learning tasks where there are many labeled training data from a \textit{source} domain, and little or no labeled testing data from a \textit{target} domain. For example, one may easily collect a lot of indoor speech records and corresponding transcripts to train a model for speech recognition, and he wants to apply the model outdoors where the background noice is different from his dataset.

Under this general setting, one fundamental question arises: "Is it possible to use training data from a source domain to learn a classifier which performs well on a target domain?" Intuitively, for effective domain transfer to be achieved, the features extracted from both training and testing data must be as indistinguishable as possible between source and target domains. In this report, we will introduce the theoretical upper bound on the true target risk established in the literature, which formalizes the above intuition theoretically. Based on these theoretical principles, we then introduce the domain-adversarial neural network (DANN) that implements this idea in the context of neural network architectures and shows decent empirical results in many domain adaptation problems.

The rest of the report is organized as follows. The domain adaptation problem is formally stated in Section \ref{sect:prob_form}. Section \ref{sect:divergence} discuss the distance measure between different domains, as a building block for the theoretical guarantee established in \ref{sect:bound}. Section \ref{sect:dann} introduce implementation, called domain-adversarial neural network (DANN), based on the main idea of theorems established in earlier sections. Finally, Section \ref{sect:conclusion} concludes this report.

\section{Problem Formulation}\label{sect:prob_form}
In this section, we formally state the domain adaptation problem. Let $\mathcal{X}$ denotes the input data space and $\mathcal{Y}$ the label space. The term \textit{domain} used in this report is defined as a pair consisting of a distribution $\mathcal{D}^X$ over $\mathcal{X}$ and a labeling function $f: \mathcal{X} \rightarrow \mathcal{Y}$. We consider two domains, a \textit{source} domain: $\DS \equiv \langle \mathcal{D}^X_S, f_S \rangle$, and a \textit{target} domain: $\DT \equiv \langle \mathcal{D}^X_T, f_T \rangle$.

In a typical domain adaptation problem, the training data consist of a set of \textit{labeled} data sampled from the source domain, and a set of \textit{unlabeled} data sampled from the target domain, which can be mathematicaly denoted as $((X_1,Y_1),(X_2,Y_2),...,(X_n,Y_n)) \sim (\DS)^{\otimes n}$ and $(X_{n+1},X_{n+2},...,X_N) \sim (\mathcal{D}^X_T)^{\otimes n'}$, respectively, with $N=n+n'$ being the total number of samples. Furthermore, we use $\US$ and $\UT$ to denote the set of (unlabeled) data $X_i$'s from the source and target domains, respectively.

% A \textit{hypothesis set} is defined as $\mathcal{H}=\{h:\mathcal{X} \rightarrow \mathcal{Y}\}$.

For theoretical reasons, we consider binary classification problems with label space $\mathcal{Y}=[0, 1]$, which can have fractional values when the labeling function (i.e., the \textit{black box}) is non-deterministic. In this case, the \textit{hypothesis set} is defined as $\mathcal{H} \subseteq \{h:\mathcal{X} \rightarrow \{0,1\}\}$. Throughout this report, we consider the \textit{symmetric} hypothesis set, in which for every $h$, the inverse hypothsis $1-h$ is also in hypothesis set. Also, the loss function $\ell: \mathcal{Y} \times \mathcal{Y} \rightarrow \mathbb{R}$ is simply defined as $\ell (\hat{y},y) \triangleq I \left[ \hat{y} \neq y \right]$, with $I[.]$ being the binary indicator variable which is $1$ if the statement is true. Hence, the \textit{statistical risk} can be defined as
\begin{align}
\RiskS(h) \triangleq \RiskS(h, f_S) = & \expect{\DS^X}{\abs{h(X) - f_S(X)}}, \label{eq:RiskS}\\
\RiskT(h) \triangleq \RiskT(h, f_T) = & \expect{\DT^X}{\abs{h(X) - f_T(X)}}, \label{eq:RiskT}
\end{align}
Accordingly, the \textit{empirical risk}, which is valid only for the source domain, can be defined as
\begin{equation}\label{eq:ERiskS}
\ERiskS(h) \triangleq \frac{1}{n} \sum_{i=1}^{n} \abs{h(X_i) - Y_i}.
\end{equation}

The goal of the learning algorithm is to find a good hypothesis $h \in \mathcal{H}$ with a low target risk $\RiskT(h)$ based on the labeled data from the source domain and unlabeled data on the target domain.


% \begin{itemize}
% \item \textbf{Data space:} $\mathcal{X} \times \mathcal{Y}$.
% \item \textbf{Testing data:} $(X,Y)$, where $X \sim \DT$ with $\DT$ being the target distribution over $\mathcal{X}$, and 

% \item \textbf{Data Distribution on} $\mathcal{X}$ \textbf{for two domains:} $\DS$, $\DT$
% \item Labeling function for two domains: $f_S$, $f_T$: $\mathcal{X} \rightarrow [0, 1]$
% \item Hypothesis Set: $\mathcal{H}=\{h:\mathcal{X} \rightarrow \{0,1\}\}$. Throughout this report, we consider the \textit{symmetric} hypothesis set, in which for every $h$, the inverse hypothsis $1-h$ is also in hypothesis set.
% \item Disagreement (risk) for two domains:
% $$\RiskS(h) \triangleq \RiskS(h, f_S) = \expect{X \sim \DS}{\abs{h(X) - f_S(X)}}$$
% $$\RiskT(h) \triangleq \RiskT(h, f_T) = \expect{X \sim \DS}{\abs{h(X) - f_T(X)}}$$
% \item Goal: To minimize the risk of the testing data drawn from a target domain under the model trained on data drawn from a source domain.
% \end{itemize}
\section{Classifier-Induced Divergence Measure}
In this section, we introduce the classifier-induced divergence measure widely used in the literature. In the following subsections, let $\mathcal{D}$ and $\mathcal{D}'$ denote two probability distributions over the same measure space $(\mathcal{X}, \mathcal{E})$, where $\mathcal{X}$ is the domain set and $\mathcal{E}$ is its measurable subsets under $\mathcal{D}$ and $\mathcal{D}'$.

\subsection{Total variation}
The most natural measure of divergence between two distributions is the \textit{total variation} distance. The total variation between $\mathcal{D}$ and $\mathcal{D}'$ is defined as
\begin{equation}\label{eq:total_var}
  d_{L_1}(\mathcal{D}, \mathcal{D}') \triangleq 2\sup_{E \in \mathcal{E}} \vert P_{\mathcal{D}}(E)-P_{\mathcal{D}'}(E) \vert.
\end{equation}
That is, the total variation measures the largest possible difference between the two probability distributions by considering all posible events.

However, the total variation is not very useful since it is too sensitive and inflates the bound unnecessary. Also, it is feasibly difficult to estimate the total variation between two distributions from finite samples they generate. As pointed out in \cite{Kifer2004}, it is often sufficient in practice to focus on a \textit{family} of significant domain subsets, which leads to the definitions in the following subsections.

\subsection{$\mathcal{A}$-distance}
Let $\mathcal{A}$ be a collection of measurable sets. The $\mathcal{A}$-distance between $\mathcal{D}$ and $\mathcal{D}'$ is defined as
\begin{equation}\label{eq:a_distance}
  d_{\mathcal{A}}(\mathcal{D}, \mathcal{D}') \triangleq 2\sup_{A \in \mathcal{A}} \vert P_{\mathcal{D}}(A)-P_{\mathcal{D}'}(A) \vert.
\end{equation}
Informally, the $\mathcal{A}$-distance is the largest possible difference between the two probability distributions by considering events that \textit{the user cares about}. For example, let the domain set be all the real numbers and $\mathcal{A}$ be the set of all one-sided intervals $(-\infty,x), x \in \mathbb{R}$. In this case, the $\mathcal{A}$-distance becomes the Kolmogorov-Smirnov statistic: $\sup_{x \in \mathbb{R}} \vert F_{\mathcal{D}}(x)-F_{\mathcal{D}'}(x) \vert$, with $F_{\mathcal{D}}$ and $F_{\mathcal{D'}}$ being cumulative distribution functions over $\mathcal{D}$ and $\mathcal{D}'$, respectively.

\subsection{$\mathcal{H}$-divergence}
We can now define the \textit{classifier-induced divergence measure} based on the $\mathcal{A}$-distance. Given the hypothesis set $\mathcal{H}$, let $I(h) \triangleq \{ \mathbf{x} \vert h(\mathbf{x})=1 \}$ denote the subset of $\mathcal{X}$ induced by the characteristic function $h \in \mathcal{H}$. The $\mathcal{H}$-divergence between two distributions $\mathcal{D}$ and $\mathcal{D}'$ is defined as
\begin{equation}\label{eq:h_divergence}
  d_{\mathcal{H}}(\mathcal{D}, \mathcal{D}') \triangleq 2\sup_{h \in \mathcal{H}} \vert P_{\mathcal{D}}(I(h))-P_{\mathcal{D}'}(I(h)) \vert.
\end{equation}
Hence, the $\mathcal{H}$-divergence can be seen as one special kind of the $\mathcal{A}$-distance focusing on domain subsets whose characteristic functions are those hypotheses $h \in \mathcal{H}$.

The $\mathcal{H}$-divergence addresses both limitations associated with the total variations. First, it is in general smaller and less sensitive than the total variation distance as long as $\mathcal{H}$ has finite VC dimension. Second, it can be estimated from finite samples drawn from the two distributions, as showed in the following lemma given in \cite{BenDavid2006}.

\begin{lemma}\label{lem:emp_h_divergence}
  For a symmetric hypothesis set $\mathcal{H}$ (one where for every $h \in \mathcal{H}$, the inverse hypothsis $1-h$ is also in $\mathcal{H}$) and samples $\mathcal{U}$ and $\mathcal{U}'$ drawn i.i.d. of size $m$ from two distributions $\mathcal{D}$ and $\mathcal{D}'$, respectively. We have
  \begin{equation}\label{eq:emp_h_divergence}
    \hat{d}_{\mathcal{H}}(\mathcal{U}, \mathcal{U}')=2 \left( 1-\min_{h \in \mathcal{H}} \left[ \frac{1}{m} \sum_{\mathbf{x}:h(\mathbf{x})=1}I[\mathbf{x} \in \mathcal{U}]+\frac{1}{m} \sum_{\mathbf{x}:h(\mathbf{x})=0}I[\mathbf{x} \in \mathcal{U}'] \right] \right),
  \end{equation}
  where $I[.]$ is the binary indicator variable which is $1$ if the statement is true.
\end{lemma}
\begin{proof}
  For any hypothesis $h \in \mathcal{H}$ and the corresponding subset $I(h)$ that contains all positively labeled instances, we have
  \begin{equation}\label{eq:emp_h_divergence_pf}
    1 - \left[ \frac{1}{m} \sum_{\mathbf{x}:h(\mathbf{x})=1}I[\mathbf{x} \in \mathcal{U}]+\frac{1}{m} \sum_{\mathbf{x}:h(\mathbf{x})=0}I[\mathbf{x} \in \mathcal{U}'] \right]
    =
    P_{\mathcal{U}}(I(h))-P_{\mathcal{U}'}(I(h)).
  \end{equation}
  The absolute value in the statement of the lemma follows from the symmetry of $\mathcal{H}$.
\end{proof}

This lemma suggests that, instead of computing the empirical $\mathcal{H}$-divergence directly by taking the supremum over $\mathcal{H}$, which is generally difficult, we can try to find a hypothesis $h \in \mathcal{H}$ which has minimum error for the binary classification problem of distinguishing the two domains. To do so, we can label all the source samples by $0$ and all the target samples by $1$, and then use the risk of the binary classier trained on the combination of source and target samples to approximate the "$\min$" part of (\ref{eq:emp_h_divergence}) in Lemma \ref{lem:emp_h_divergence}.

\subsection{$\HDH$-divergence}
Finally, the idea of $\mathcal{H}$-divergence can be extended to the \textit{symmetric difference hypothesis space} defined as $\HDH \triangleq \{ h(\mathbf{x}) \text{ XOR } h'(\mathbf{x}) \vert h, h' \in \mathcal{H} \}$, which leads to the difinition of the $\HDH$-divergence as
\begin{equation}\label{eq:hdh_divergence}
  \DHDH(\mathcal{D}, \mathcal{D}') \triangleq 2\sup_{h \in \HDH} \vert P_{\mathcal{D}}(I(h))-P_{\mathcal{D}'}(I(h)) \vert.
\end{equation}
Note that the $\HDH$-divergence can also be written as
\begin{equation}\label{eq:hdh_divergence2}
  2\sup_{h, h' \in \mathcal{H}} \vert P_{\mathcal{D}}(h(\mathbf{x}) \neq h'(\mathbf{x}))-P_{\mathcal{D}'}(h(\mathbf{x}) \neq h'(\mathbf{x})) \vert,
\end{equation}
which measures the largest possible difference between the two probability distributions when considering all possible \textit{disagreement} events induced by hypotheses $h,h' \in \mathcal{H}$.

% \subsection{Empirical $\mathcal{H}$-divergence}

(Toy example for $\mathcal{X}$, $\mathcal{H}$, $\mathcal{H}$-divergence, $\HDH$-divergence, the empirical $\mathcal{H}$-divergence, and the lemma about the calculation of the empirical $\mathcal{H}$-divergence)
\section{Theoretical Guarantee}
In this section, we state the main theorem and some key lemmas given in \cite{BenDavid2006} and \cite{BenDavid2010} used to establish the theoretical upper bound on the true target risk in the domain adaptation problem.

\subsection{Upper Bound on Target Risk}
\begin{theorem}[Upper Bound on Target Risk]\label{thm:A}
  Let $\mathcal{H}$ be a symmetric hypothesis set with VC dimension $d$, and let $\US$ and $\UT$ denote the samples of size $m$ drawn i.i.d. from the source and target distributions, respectively. Then, for any $\delta \in (0,1)$, with probability at least $1-\delta$ (over the choice of the samples), for every $h \in \mathcal{H}$, we have
  \begin{equation}\label{eq:A}
    \RiskT(h) \leq \RiskS(h) + \frac{1}{2} \EDHDH(\US, \UT) + 4 \sqrt{\frac{2d \log{2m} + \log{\frac{2}{\delta}}}{m}} + \epsilon^{*},
  \end{equation}
  where $\epsilon^{*} = \RiskS(h^*) + \RiskT(h^*)$ and $h^* = \argmin_{h \in \mathcal{H}} \RiskS(h) + \RiskT(h)$.
\end{theorem}

The bound in Theorem \ref{thm:A} is related to the optimal combined error $\epsilon^{*}$ and the corresponding optimal classifier $h^*$ in the hypothesis set. When the optimal combined error is large, it means that there is no classifier that performs well on both source and target domains, and hence it is generally impractical to seek to a good target hypothesis by training only on the source domain. As a result, it is often assumed in the literature that $\epsilon^{*}$ is small, which is the case in most domain adaptation problems with reasonable representation function $\mathcal{R}$.

To prove Theorem \ref{thm:A}, we need the following lemmas.

\begin{lemma}[Triangle Inequality]\label{lem:triangle}
  For any labeling functions $f_1$, $f_2$, and $f_3$, we have
  \begin{equation}\label{eq:triangle}
    \epsilon(f_1,f_2) \leq \epsilon(f_1,f_3) + \epsilon(f_2,f_3).
  \end{equation}
\end{lemma}
\begin{proof}
  By definition,
  \begin{align}
    \epsilon(f_1, f_2)
    =   & \expect{X}{\abs{f_1(X)-f_2(X)}} \label{eq:triangle_1}\\
    \leq& \expect{X}{\abs{f_1(X)-f_3(X)} + \abs{f_2(X)-f_3(X)} } \label{eq:triangle_2}\\
    =   & \expect{X}{\abs{f_1(X)-f_3(X)}} + \expect{X}{\abs{f_2(X)-f_3(X)}} \label{eq:triangle_3}\\
    =   & \epsilon(f_1, f_3) + \epsilon(f_2, f_3). \label{eq:triangle_4}
  \end{align}
\end{proof}

\begin{lemma}[Hypothesis Error Bound]\label{lem:h_bound}
  For any hypotheses $h, h' \in \mathcal{H}$, we have
  \begin{equation}\label{eq:h_bound}
    \vert \RiskS(h,h') - \RiskT(h,h') \vert
    \leq
    \frac{1}{2} \DHDH(\DS,\DT).
  \end{equation}
\end{lemma}
\begin{proof}
  \begin{align}
    \DHDH(\DS, \DT)
    =& 2 \sup_{h \in \HDH} \abs{P_{X \sim \DS}(I[h(X)]) - P_{X \sim \DT}(I[h(X)])} \label{eq:h_bound_1}\\
    =& 2 \sup_{h, h' \in \mathcal{H}} \abs{P_{X \sim \DS}(I[h(X) \ne h'(X)]) - P_{X \sim \DT}(I[h(X) \ne h'(X)])} \label{eq:h_bound_2}\\
    =& 2 \sup_{h, h' \in \mathcal{H}} \abs{\RiskS(h, h') - \RiskT(h, h')} \label{eq:h_bound_3}\\
    \geq& 2 \abs{\RiskS(h, h') - \RiskT(h, h')}. \label{eq:h_bound_4}
  \end{align}
\end{proof}

\begin{lemma}[Uniform Convergence of $\HDH$-divergence]\label{lem:unif_conv_h}
  Let $\mathcal{H}$ be a hypothesis set on $\mathcal{X}$ with VC dimension $d$, and let $\US$ and $\UT$ be samples of size $m$ drawn i.i.d. from $\DS$ and $\DT$, respectively. Then, for any $\delta \in (0,1)$, with probability at least $1-\delta$, we have
  \begin{equation}\label{eq:unif_conv_h}
    \DHDH(\DS, \DT)
    \leq
    \EDHDH(\US, \UT)
    +
    4 \sqrt{\frac{2d \log{2m} + \log{\frac{2}{\delta}}}{m}}.
  \end{equation}
\end{lemma}
\begin{proof}
\end{proof}

We can now prove Theorem \ref{thm:A} as follows.

\begin{proof}[Proof of Theorem \ref{thm:A}]
  \begin{align}
     \RiskT(h) 
    =& \RiskT(h, f_T) \label{eq:A_1}\\
    \leq& \RiskT(h^*, f_T) + \RiskT(h, h^*) \label{eq:A_2}\\
    \leq& \RiskT(h^*, f_T) + \RiskS(h, h^*) + \abs{\RiskT(h, h^*) - \RiskS(h, h^*)} \label{eq:A_3}\\
    \leq& \RiskT(h^*, f_T) + \RiskS(h, h^*) + \frac{1}{2} \DHDH(\DS, \DT) \label{eq:A_4}\\
    \leq& \RiskT(h^*, f_T) + \RiskS(h, f_S) + \RiskS(h^*, f_S) + \frac{1}{2} \DHDH(\DS, \DT) \label{eq:A_5}\\
    =   & \RiskS(h, f_T) + \epsilon^* + \frac{1}{2} \DHDH(\DS, \DT) \text{\quad (by definition)} \label{eq:A_6}\\
    \leq& \RiskS(h, f_T) + \epsilon^* + \frac{1}{2} \EDHDH(\US, \UT)
          + 4 \sqrt{\frac{2d \log(2m) + \log(\frac{2}{\delta})}{m}}, \label{eq:A_7}
  \end{align}
  where (\ref{eq:A_2}) and (\ref{eq:A_5}) are by Lemma \ref{lem:triangle}, (\ref{eq:A_4}) is by Lemma \ref{lem:h_bound}, and (\ref{eq:A_7}) is by Lemma \ref{lem:unif_conv_h}.
\end{proof}

To further bound the true source risk by its empirical version, let $m'$ denotes the size of \textit{labeled} data drawn i.i.d. from the source domain. Then, by applying the VC bound, for any $\delta \in (0,1)$, with probability at least $1 - \delta$, we have
\begin{equation}\label{eq:risk_s_vc}
  \RiskS(h) \leq \ERiskS(h) + \sqrt{\frac{4}{m'}\left( d \log \frac{2em'}{d} + \log \frac{4}{\delta} \right)}.
\end{equation}

By plugging-in (\ref{eq:risk_s_vc}) into Theorem \ref{thm:A}, we obtain the following corollary.

\begin{corollary}\label{thm:B}
  For any $\delta \in (0,1)$, with probability at least $1 - \delta$, we have
  \begin{equation}\label{eq:B}
    \begin{aligned}
      \RiskT(h) \leq&
      \ERiskS(h) + \sqrt{\frac{4}{m'}\left( d \log \frac{2em'}{d} + \log \frac{4}{\delta} \right)} \\
      & + \frac{1}{2} \EDHDH(\US, \UT)
      + 4 \sqrt{\frac{2d \log(2m) + \log(\frac{2}{\delta})}{m}} 
      + \epsilon^*
    \end{aligned}
  \end{equation}
\end{corollary}

We can focus on the first term $\ERiskS(h)$ and the third term $\frac{1}{2} \EDHDH(\US, \UT)$, with the former being the empirical source risk and the latter being the empirical divergence between the source and target domains. Ideally, we want both terms to be small after training. However, it turns out that we can only try to minimize a tradeoff between them, as pointed out first by \cite{BenDavid2006} and implemented by \cite{Ganin2016}.

\section{Empirical Results}\label{sect:dann}
In this section, we introduce the domain-adversarial neural network (DANN) proposed by Ganin et al. \cite{Ganin2016}, \cite{Ganin2015} and Ajakan et al. \cite{Ajakan2014}.

\subsection{Model Architecture}\label{sub:model_arch}
\begin{figure}
  \centering
  \input{figures/dann-arch.tikz}
  \caption{The high-level architecture of DANN}
  \label{fig:dann_arch}
\end{figure}

The high-level architecture of DANN consists of three components: the feature extractor $G_f$; the label classifier $G_y$; and the domain classifier $G_d$, as shown in Figure \ref{fig:dann_arch}. We use $\theta_f$, $\theta_y$, and $\theta_d$, respectively, to denote the set of parameters of each compunent.

As can be seen from Figure \ref{fig:dann_arch}, the feature extractor $G_f$ takes the role of the representation function $\mathcal{R}$, which transforms the input data from the domain space $\mathcal{X}$ into a feature space, say, $\mathcal{Z}$. Hence, at the output of the feature extractor, we will get two distributions $\tDS$ and $\tDT$ over $\mathcal{Z}$, each of which induced from $\DS$ and $\DT$ over $\mathcal{X}$, respectively. The label classifier $G_y$ and the domain classifier $G_d$ then take the transformed data $Z=G_f(X;\theta_f)$ as their input. Intuitively, a good representation must be discriminative for $G_y$ with respect to the main learning task on the source domain, while at the same time be indiscriminative for $G_d$ with respect to the shift between the source and target domains. If it is the case, we then have more condifence to say that: If $G_y$ can be trained on the source domain to do well on $\tDS$, then it should also do well on $\tDT$.

\subsection{Loss and Objective Functions}
% Without loss of generality, consider the neural network with the single layer setting. (parameters, loss functions, the objective function, ...)

Let the loss functions of the label classifier $G_y$ and the domain classifier $G_d$ be denoted by
\begin{equation}\label{eq:loss_y}
  L_y(X, Y; \theta_f, \theta_y) \triangleq \ell (G_y(G_f(X; \theta_f); \theta_y), Y)
\end{equation}
and
\begin{equation}\label{eq:loss_d}
  L_d(X, D; \theta_f, \theta_d) \triangleq \ell (G_d(G_f(X; \theta_f); \theta_d), D),
\end{equation}
respectively. According to the intuition about a good representation mentioned in Section \ref{sub:model_arch}, the model must be trained to minimize $L_y(X, Y; \theta_f, \theta_y)$, and at the same time maximize $L_d(X, D; \theta_f, \theta_d)$. Training DANN is thus equivalent to optimizing the objective function
\begin{equation}\label{eq:obj}
  \begin{aligned}
    E(\theta_f, \theta_y, \theta_d)
    =& \underbrace{\frac{1}{n} \sum_{i=1}^{n} L_y(X_i, Y_i; \theta_f, \theta_y)}
       _{\text{label loss}} \\
     & - \lambda \underbrace{(
       \frac{1}{n} \sum_{i=1}^{n} L_d(X_i, S; \theta_f, \theta_d)
       + \frac{1}{n'} \sum_{i=n+1}^{n + n'} L_d(X_i, T; \theta_f, \theta_d))}
       _{\text{domain loss}}
  \end{aligned}
\end{equation}
by finding the \textit{saddle} point $\hat{\theta}_f$, $\hat{\theta}_y$, and $\hat{\theta}_d$ such that
\begin{align}
  (\hat{\theta}_f, \hat{\theta}_y) = & \argmin_{\theta_f, \theta_y} E(\theta_f, \theta_y, \hat{\theta}_d), \label{eq:obj_argmin}\\
  \hat{\theta}_d = & \argmax_{\theta_d} E(\hat{\theta}_f, \hat{\theta}_y, \theta_d). \label{eq:obj_argmax}
\end{align}

In other words, the feature extractor $G_f$ and the label classifier $G_y$ are trained to minimize the objective function by updating $\theta_f$ and $\theta_y$, while the domain classifier $G_d$ is trained to maximize it by updating $\theta_d$. Note that the domain classifier and the feature extractor are trained \textit{adversarially} against each other, in a way that is similar to the minimax game between generator and discriminator in the generative adversarial network (GAN) frameworks, which suggest the name of DANN.

% At training time, in order to get discriminative representations for the label classification task, we seek to minimize the loss of the label predictor $G_y$ by updating $\theta_f$ and $\theta_y$.
% On the other hand, we also want indiscriminative representations for the domain classification task. In order to get a good \textit{proxy} to the loss of the domain classifier.

\subsection{Interpretation}
How does optimizing the objective function (\ref{eq:obj}) relate to minimizing the theoretical bound? We will show that minimizing (\ref{eq:obj}) with respective to $\theta_f$ and $\theta_y$ is equivalent to minimize the first term and the third term in (\ref{eq:B}) of Corollary \ref{thm:B}. To recap, we can reformulate (\ref{eq:B}) as
\begin{equation}
  \RiskT(h) \leq \ERiskS(h) + \frac{1}{2} \EDHDH(\US, \UT) + C,
\end{equation}
where $C$ is the aggregated term related to number of samples and the hypothesis set. Clearly, $\ERiskS(h)$ is equivalent to the \textit{label loss} in (\ref{eq:obj}): $\frac{1}{n} \sum_{i=1}^{n} L_y(X_i, Y_i; \theta_f, \theta_y)$. It then sufficient to relate $\EDHDH(\US, \UT)$ to the \textit{domain loss} in (\ref{eq:obj}), denoted by

\begin{align}
\hat{L}_{d}(X, D; \theta_f, \theta_d) \triangleq &
\frac{1}{n} \sum_{i=1}^{n} L_d(X_i, S; \theta_f, \theta_d)
+
\frac{1}{n'} \sum_{i=n+1}^{n + n'} L_d(X_i, T; \theta_f, \theta_d) \label{eq:domain_loss}\\
= & \frac{1}{m} \sum_{Z:G_d(Z;\theta_d)=1}I[Z \in \TUS]+\frac{1}{m} \sum_{Z:G_d(Z;\theta_d)=0}I[Z \in \TUT], \label{eq:domain_loss_I}
\end{align}
where in (\ref{eq:domain_loss_I}) we have assumed that the unlabeled datasets have size $\abs{\TUS}=\abs{\TUT}=m$, and the source samples have domain labels $S=0$ while the target samples have domain labels $T=1$. By noting the minus sign before the domain loss in (\ref{eq:obj}), it can be seen that the maximizer $\hat{\theta}_d$ for (\ref{eq:obj_argmax}) is actually the minimizer for $\hat{L}_{d}(X, D; \theta_f, \theta_d)$. Hence, we have
\begin{align}
\hat{\theta}_d
= & \argmin_{\theta_d} \hat{L}_{d}(X, D; \theta_f, \theta_d) \label{eq:argmin_theta_d}\\
= & \argmin_{\theta_d} \left[ \frac{1}{m} \sum_{Z:G_d(Z;\theta_d)=1}I[Z \in \TUS]+\frac{1}{m} \sum_{Z:G_d(Z;\theta_d)=0}I[Z \in \TUT] \right]. \label{eq:argmin_theta_d_I}
\end{align}
Also, the minimizer $\hat{\theta}_f$ for (\ref{eq:obj_argmin}) is actually the maximizer for $\hat{L}_{d}(X, D; \theta_f, \hat{\theta}_d)$ for a fixed $\hat{\theta}_d$. We thus have
\begin{align}
\max_{\theta_f}{\hat{L}_{d}(X, D; \theta_f, \hat{\theta}_d)}
=& \min_{\theta_f}{- \hat{L}_{d}(X, D; \theta_f, \hat{\theta}_d)} \label{eq:min_theta_f}\\
=& \min_{\theta_f} 2 \left( 1 - \frac{1}{m} \min_{\theta_d} \left[ \sum_{Z:G_d(Z;\theta_d)=1}I[Z \in \TUS]+ \sum_{Z:G_d(Z;\theta_d)=0}I[Z \in \TUT] \right] \right) \label{eq:min_theta_f_I}\\
=& \min_{\theta_f} \hat{d}_{\mathcal{H}_d}(\TUS, \TUT) \label{eq:min_theta_f_EDHDH},
\end{align}
where (\ref{eq:min_theta_f_I}) is by (\ref{eq:argmin_theta_d_I}), and (\ref{eq:min_theta_f_EDHDH}) is by Lemma \ref{lem:emp_h_divergence_I} with $\mathcal{H}_d$ being the hypothesis set for the domain classifier $G_d$. As noted in \cite{Ganin2015}, we can assume that the family of domain classifier $\mathcal{H}_d$ is rich enough to contain the symmetric difference hypothesis set of the label classifier $G_y$, that is
\begin{equation}
 \mathcal{H}_y \Delta \mathcal{H}_y = \{ h(X) \text{ XOR } h'(X) \vert h, h' \in \mathcal{H}_y \}.
\end{equation}
Since we are considering neural networks, it is not an unrealistic assumption as we have a freedom to build a complex model for $G_d$ by concatenating two replicas of $G_y$ followed by a two-layer non-linear perceptron for the XOR function.

To sum up, we have showed that minimizing the objective function in DANN is equivalent to minimize $\ERiskS(h)$ and $\frac{1}{2} \EDHDH(\US, \UT)$, which form the upper bound of the true target risk $\RiskT(h)$.

% (R(W;b) ...)

% (Training algorithm ...)


\section{Conclusions}
After stating the problem setting of domain adaptation, We discuss about the target risk under unsupervised domain adaptation from theoretical viewpoint.
In the case of lacking target labels, $\HDH$-divergence is introduced to  measure the transferability between source and target domain.
Based on the theoretical result, a model called DANN is designed for unsupervised domain adaptation, which is an interesting example as an application of theoretical analysis.


\bibliographystyle{ieeetr}
\bibliography{Ref}


\end{document}
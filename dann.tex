\section{Empirical Results}
In this section, we introduce the domain-adversarial neural network (DANN) proposed by Ganin et al. \cite{Ganin2016}, \cite{Ganin2015} and Ajakan et al. \cite{Ajakan2014}.

\subsection{Model Architecture}
\begin{figure}
  \centering
    \begin{tikzpicture}
    \node (x) {$X$};
    \node[rectangle, draw, text width=1.5cm, text centered, right=1.5cm of x] (gf) {$\color{violet} G_f$};
    \node[rectangle, draw, text width=1.5cm, text centered, below right=0.5cm and 1.5cm of gf] (gd) {$\color{red} G_d$};
    \node[rectangle, draw, text width=1.5cm, text centered, above right=0.5cm and 1.5cm of gf] (gy) {$\color{blue} G_y$};
    \node[right=2cm of gd] (d) {$\{ \DS, \DT \}$};
    \node[right=2cm of gy] (y) {$Y$};    
    % \draw (7,5) rectangle (9, 6) node[pos=.5] {$G_y$};
    \path [draw, -latex'] (x) -- (gf);
    \path [draw, -latex'] (gf.east) -- (gy.west);
    \path [draw, -latex'] (gf.east) -- (gd.west);
    \path [draw, -latex'] (gy) -- (y);
    \path [draw, -latex'] (gd) -- (d);
  \end{tikzpicture}
  \caption{The high-level architecture of DANN}
  \label{fig:dann_arch}
\end{figure}

The high-level architecture of DANN consists of three components: the feature extractor $G_f$; the label classifier $G_y$; and the domain classifier $G_d$, as shown in Figure \ref{fig:dann_arch}. We use $\theta_f$, $\theta_y$, and $\theta_d$, respectively, to denote the set of parameters of each compunent.

As can be seen from Figure \ref{fig:dann_arch} the feature extractor $G_f$ takes the role of the representation function $\mathcal{R}$, which transforms the input data from the domain space $\mathcal{X}$ into a feature space, say, $\mathcal{Z}$. Hence, at the output of the feature extractor, we will get two distributions $\tDS$ and $\tDT$ over $\mathcal{Z}$, each of which induced from $\DS$ and $\DT$ over $\mathcal{X}$, respectively. The label classifier $G_y$ and the domain classifier $G_d$ then take the transformed data $\mathbf{z}=G_f(\mathbf{x};\theta_f)$ as input. Intuitively, a good representation must be discriminative for $G_y$ with respect to the main learning task on the source domain, while at the same time be indiscriminate for $G_d$ with respect to the shift between the source and target domains. If it is the case, we then have more condifence to say that, if $G_y$ can be trained on the source domain to do well on $\tDS$, then it should also do well on $\tDT$.

\subsection{Loss and Objective Functions}
Without loss of generality, consider the neural network with the single layer setting. (parameters, loss functions, the objective function, ...)

Let the loss functions of the label classifier $G_y$ and the domain classifier $G_d$ be denoted by
\begin{equation}\label{eq:loss_y}
  L_y(X, Y; \theta_f, \theta_y) = l(G_y(G_f(X; \theta_f); \theta_y), Y)
\end{equation}
and
\begin{equation}\label{eq:loss_d}
  L_d(X, D; \theta_f, \theta_d) = l(G_y(G_f(X; \theta_f); \theta_d), D),
\end{equation}
respectively. According to the intuition about a good representation mentioned earlier, the model must be trained to minimize $L_y(X, Y; \theta_f, \theta_y)$, and at the same time to maximize $L_d(X, D; \theta_f, \theta_d)$. Training DANN is thus equivalent to optimizing the objective function
\begin{equation}\label{eq:obj}
  \begin{aligned}
    E(\theta_f, \theta_y, \theta_d)
    =& \underbrace{\frac{1}{n} \sum_{i=1}^{n}  L_y(X_i, Y_i; \theta_f, \theta_y)}
       _{\text{label loss}} \\
     & - \lambda \underbrace{(
       \frac{1}{n} \sum_{i=1}^{n} L_d(X_i, S; \theta_f, \theta_d)
       + \frac{1}{n'} \sum_{i=n+1}^{n + n'} L_d(X_i, T; \theta_f, \theta_{d}))}
       _{\text{domain loss}}
  \end{aligned}
\end{equation}
by finding the \textit{saddle} point $\hat{\theta}_f$, $\hat{\theta}_y$, and $\hat{\theta}_d$ such that
\begin{align}
  (\hat{\theta}_f, \hat{\theta}_y) = & \argmin_{\theta_f, \theta_y} E(\theta_f, \theta_y, \hat{\theta}_d), \label{eq:opt_argmin}\\
  \hat{\theta}_d = & \argmax_{\theta_d} E(\hat{\theta}_f, \hat{\theta}_y, \theta_d). \label{eq:opt_argmax}
\end{align}

In other words, the domain classifier and the feature extractor are trained \textit{adversarially} againest each other, in a way that is similar in the minimax game between generator and discriminator in the generative adversarial network (GAN) frameworks.

% At training time, in order to get discriminative representations for the label classification task, we seek to minimize the loss of the label predictor $G_y$ by updating $\theta_f$ and $\theta_y$.
% On the other hand, we also want indiscriminative representations for the domain classification task. In order to get a good \textit{proxy} to the loss of the domain classifier.

\subsection{Interpretation}
How does optimizing the objective function (\ref{eq:obj}) relate to minimizing the theoretical bound? We will show that minimizing (\ref{eq:obj}) with respective to $\theta_f$ and $\theta_y$ is equivalent to minimize the first term and the third term in (\ref{eq:B}) of Corollary \ref{thm:B}.

(R(W;b) ...)

(Training algorithm ...)

